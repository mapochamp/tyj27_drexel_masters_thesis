% Chapter Template

\chapter{OnSRAM} % Main chapter title

\label{Chapter3} % Change X to a consecutive number; for referencing this chapter elsewhere, use \ref{ChapterX}

%----------------------------------------------------------------------------------------
%	SECTION 1
%----------------------------------------------------------------------------------------

\section{Introduction}
% do we need to talk about all the new accels that come out and why they are needed to run DL workloads instead of CPUs?

OnSRAM introduces the notion of two types of runtime performance optimizations
that can be done on graph based compilers: intra-node and inter-node
optimizations \cite{onsram}.  Intra node optimizations are those that are
focused on optimizing the operation that the node specifies. This means tiling
in favorable sizes and across specific dimensions, loop ordering, and DMA
pipelining between tiles \cite{aladdin}.

Inter node optimizations are optimizations concerning the overall structure and
ordering of the nodes in the graph and their connections. This includes
operator fusion and node reordering or rescheduling \cite{onsram}.
OnSRAM focuses on internode optimizations, specifically in the domain of
scratchpad memory management techniques for DLAs. 

Existing frameworks have neglected to focus on inter-node optimizations in favor
of intra-node optimizations \cite{tvm} \cite{deeptools} \cite{tensorflow}. Many 
mentioned frameworks are focused on creating an overall compiler framework that
increases the usability and hardware abstractions for machine learning engineers
by enabling hardware agnostic support and optimizations. As a consequence, each
framework had taken a different approach or extended the capabilities to of other
existing frameworks to achieve the same goal. In comparison, OnSRAM is a
compiler extension made to focus on an algorithm specific to just one type of
memory optimization.

The motivation of such optimizations come from the minimization of memory
transfers which speed up inference runs by up to 5.17x \cite{onsram}. OnSRAM
exploits the repeating usage of outputs in a graph and identifies the outputs
that can pinned to minimize memory transfers which decrease inference time and
decrease energy costs. The ultimate performance gained from pinning outputs comes from
multiple inference runs where the cost of mapping pinnable tensors are amortized by the
time saved through iterative inference runs of the same model.

Such inter-node memory management techniques occur due to the presence of
inherent repeating patterns of operations to support deep learning
architectures, including matrix multiplication and convolution operations.


%-----------------------------------
%	SUBSECTION 1
%-----------------------------------

\subsection{Static Graph Execution}

OnSRAM-Static, the static DNN Graph SPM manager is described as follows. A
graph of operations that represent a DNN is passed to OnSRAM-Static as an
input. The vertexes of the graph are the operations to be executed, e.g matrix
multiply, and the edges represent input and output tensors.  All input and
output data tensors of each node is analyzed for the start time, end time,
number of reuses, distance between reuses, and size of the tensor. These
properties are then used to determine the optmial pinning scheme to minimize
memory transfers. Each of these properties are considered with weights on how
important each property is and the tensors are psuedo-sorted in cost order;
they are psuedo-sorted since tensors that don't overlap in their lifetimes need
not be sorted relative to each other. These sorted tensors are then considered
in a greedy fashion where tensors of the highest cost are considered for
pinning first and the rest are considered to be pinned if and only if it can be
accommodated in spite of the higher cost tensor being pinned. Tensors are only
considered for pinning if they can be pinned for their entire duration of their
lifetime and can fit on the SPM with the other inputs and outputs for every
operation they are pinned.

OnSRAM-Static only applies to input and output tensors, not weights. This is
due to the fact that weights are single use objects that are only needed for
the layers they are used in. There exists no reuse opportunity unlike output
tensors since all layers contain independently different weights. Thus, weights
are not considered for pinning for both OnSRAM-Static and our extension work as
well.

Another aspect that OnSRAM does not consider is applicability towards training.
The assumptions of the problem change when considering training and
significantly increase the difficulty and search space. This requires the
separate compiler extension implementations to be designed into separate
portions of the framework. Further, batched training, partitioned accelerators,
distributed systems, and large reuse distances due to backward passes and
batched data \cite{onsram} bring significant design challenges that are not present in
inference. Thus we have also chosen not to extend this work in this regard as
well.

%----------------------------------------------------------------------------------------
%	SECTION 2
%----------------------------------------------------------------------------------------

\section{Simulation and Architecture}
% how important is this question? should we just be saying what we used

Differences in hardware and simulators don't affect the relative performance of
a non-pinning compiler compared to one that has an optimized management scheme,
however there are some important environment variables that will consequently
affect repeatability and benchmark performance. Most notably, the simulator and
hardware architecture in use.

OnSRAM uses a cycle accurate simulator to model and assess the performance of
their algorithm \cite{onsram}.  We have opted to use the SMAUG\cite{smaug}
framework that depends on the Aladdin\cite{aladdin} system to model our
accelerator and the gem5 simulator\cite{gem5} to implement our experiments.
While gem5 is not cycle accurate, it still advertises a close to cycle accurate
performance metrics and creates a relative base of comparison between benchmark
models and our optimized model.

Intra-node optimizations that both simulators do but may not do the same way
that may affect performance comparisons: loop tiling, loop ordering, unrolling,
and pipelined DMA operations for maximizing intra-node reuse.

% architectural differences
We use the default SMAUG provided accelerator inspired by NVDLA\cite{smaug}.
The accelerator contains eight PEs, multiply accumalate (MACC) arrays, and
contains 3 scratchpad memories\cite{smaug}. Two scratchpads are configured to
be used as inputs and one for output.

% do we need to specify onsram architecture?

%----------------------------------------------------------------------------------------
%	SECTION 3
%----------------------------------------------------------------------------------------

\section{Extensions}

%-----------------------------------
%	SUBSECTION 1
%-----------------------------------

\subsection{OnSRAM Limitations}

While OnSRAM has contributed a novel SPM management scheme that no framework
currently supports \cite{onsram}, there are limitations that don't allow it to
generalize to a hardware agnostic scenario. This is primarily due to its
assumption that the accelerator contains only one shared scratchpad for input
and output tensors. A consequence of this is that the greedy algorithm proposed
to map possible pin candidates is non-trivial to apply to an accelerator with
more than one scratchpad. Further, because of the heuristic approach, OnSRAM is not able
to achieve the optimal memory mappings with respect to minimizing memory transfer.

%-----------------------------------
%	SUBSECTION 2
%-----------------------------------

\subsection{Motivating Example}

%-----------------------------------
%	SUBSECTION 3
%-----------------------------------

\subsection{Problems with Generalizing a Single SPM Management Scheme to Multiple SPMs}
\subsubsection{Attempted Greedy Approaches}
\subsubsection{Attempted Architectural Assumptions}
