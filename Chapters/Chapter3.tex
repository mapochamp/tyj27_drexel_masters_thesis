% Chapter Template

\chapter{OnSRAM} % Main chapter title

\label{Chapter3} % Change X to a consecutive number; for referencing this chapter elsewhere, use \ref{ChapterX}

%----------------------------------------------------------------------------------------
%	SECTION 1
%----------------------------------------------------------------------------------------

\section{Introduction}
% do we need to talk about all the new accels that come out and why they are needed to run DL workloads instead of CPUs?

OnSRAM introduces the notion of two types of optimizations that can be done on
graph based compilers: intra-node and inter-node optimizations \cite{onsram}.
Intra node optimizations are those that are focused on optimizing the operation
that the node specifies. This means tiling in favorable sizes and across
specific dimensions, loop ordering, and DMA pipelining
between tiles \cite{Aladdin}.
Inter node optimizations are optimizations concerning the overall structure and
ordering of the nodes in the graph and their connections. This includes
operator fusion and node reordering or rescheduling \cite{onsram}.
OnSRAM focuses on internode optimizations, specifically in the domain of
scratchpad memory management techniques for DLAs. OnSRAM exploits the 
repeating useage of nodes in a graph -> some more detail on how graphs look like and why they're repeateable <-
and identifies the nodes that can pinned to minimize memory transfers which increase inference time and decrease energy costs.


%-----------------------------------
%	SUBSECTION 1
%-----------------------------------
\subsection{Subsection 1}


%-----------------------------------
%	SUBSECTION 2
%-----------------------------------

\subsection{Subsection 2}


%----------------------------------------------------------------------------------------
%	SECTION 2
%----------------------------------------------------------------------------------------

\section{Generalizing To Multiple Scratchpads}

